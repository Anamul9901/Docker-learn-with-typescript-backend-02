1: create a network
---> docker network create ts-docker-net

------------------Database dockeraz--------------------
1: docker pull mongo
2: docker run --name mongodb --rm -v ts-docker-db:/data/db --network ts-docker-net mongo
["-v ts-docker-db:/data/db" ai named volume er maddhome mongodb e 'ts-docker-db' file e backend er data save kore rakhbe. tai mongodb container delete kore felleo data save thakbe]


------------------Backend dockeraz---------------------
1: docker build -t ts-docker-backend:v1 .
2: docker run --name ts-backend-container -p 5000:5000 --rm --network ts-docker-net --env-file .env -w //app -v ts-docker-logs://app/logs -v "//$(pwd)"://app -v //app/node_modules ts-docker-backend:v1

---------------Frontend Dockeriz(for frontend code)--------------------
1: {
    # NEXT_PUBLIC_API_URL=http://localhost:5000
    NEXT_PUBLIC_API_URL=http://ts-backend-container:5000
    # local host dele container er frontend, backend er sathe connect krte parbe nah.
    # karon frontend locally run hoche nah, container er network e ran hoche
    # localhost er poriborte backend container er name dete hobe
    }
2: docker build -t ts-docker-frontend:v1 .
3: docker run --name ts-docker-frontend-container --rm -p 3000:3000 --env-file .env.local -w //app -v "//$(pwd)"://app -v //app/node_modules -e WATCHPACK_POLLING=true --network ts-docker-net ts-docker-frontend:v2
[ "-e WATCHPACK_POLLING=true" aita use krle code chane krle auto update hobe browser e. aita comment e use na kore .env.local file e o use krte parbo]
3: 